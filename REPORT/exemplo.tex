\documentclass[12pt,letterpaper]{article}

\usepackage[portuguese]{babel} 
\usepackage[utf8x]{inputenc}       
\usepackage{indentfirst}
\usepackage{geometry}
\usepackage{ucs}
\usepackage{amsmath}      % Los paquetes ams son desarrollados por la American Mathematical
\usepackage{amsfonts}     % Fórmulas Matematicas
\usepackage{amssymb}
\usepackage{graphicx}     % Para insertar gráficas
\usepackage[lofdepth,lotdepth]{subfig}	% Para colocar varias figuras
\usepackage{unitsdef}	  % Para la presentación correcta de unidades
\usepackage{float}		
\usepackage{booktabs}
\usepackage{lastpage}
\usepackage{fancyhdr}	
\geometry{left=18mm,right=18mm,top=21mm,bottom=21mm} % Tamaño área de escritura de la página
\renewcommand{\unitvaluesep}{\hspace*{4pt}}	% Redimensionamiento del espacio entre magnitud 
\batchmode
\bibliographystyle{plain} 
\pagestyle{plain} 
\pagenumbering{arabic}
\pagestyle{fancy}	
\usepackage{lmodern}
\newcommand*{\escape}[1]{\texttt{\textbackslash#1}}
\newcommand*{\escapeI}[1]{\texttt{\expandafter\string\csname #1\endcsname}}
\newcommand*{\escapeII}[1]{\texttt{\char`\\#1}}

%-----------------------------------------------------------------------------------------%
%AUTORES
\author{
Mariana Medeiros\\ \texttt{A61041}\\ \texttt{e-mail}\\ \texttt{url}\\ \texttt{afilicação}\\
  \and
Pedro Faria\\ \texttt{A6009X}\\ \texttt{e-mail}\\ \texttt{url}\\ \texttt{afilicação}\\
  \and
Miguel Pinto\\ \texttt{A61049}\\ \texttt{e-mail}\\ \texttt{url}\\ \texttt{afilicação}
  \vspace*{2.0in}
}

%TITULO E SUBTITULO
\title{ Universidade do Minho\\
    \vspace*{0.60in}
    Título\\
    {\small Sub-Título}
    \vspace*{1.55in}}
      
%DATA
\date{04/32/1232}


%AMBIENTE PARA RESUMO
\newenvironment{resumo}
{	
	\begin{center}
	\begin{minipage}[t]{500 pt}
	\vspace{5mm}
	\emph{\textbf{Resumo}}
	\\[-2mm]
	\line(1,0){500}
	\\[-4.25 mm]
	\line(1,0){500}
	\\
}
{
	\normalsize
	\\[2mm]
	\footnotesize\textbf{Palavras-Chave: \footnotesize\@palabras}
	\\[-2mm]
	\line(1,0){500}
	\\[0.5cm]
	\end{minipage}
	\end{center}
}
\def\palabras#1{\gdef\@palabras{#1}}
\palabras{Palavras.}


%CABEÇALHO
\lhead{Título - SubTítulo}
\rhead{Instituição}
\cfoot{\thepage\ de \pageref{LastPage}}




%---------------------------------------------------------------------------------------%
\begin{document}

\maketitle	

\newpage
\tableofcontents

\newpage
\listoffigures

\newpage
\listoftables

\newpage
\begin{resumo}
O presente relatório descreve todo o processo de desenvolvimento, tomadas de decisão e resultados obtidos na resolução do primeiro projeto prático proposto no âmbito da Unidade Curricular de Processamento de Linguagens.
O enunciado aqui resolvido aborda o tema referente ao processamento de informação extraída da Wikipédia - variante 1. O produto final obtido consiste num processador de ficheiros .xml, descarregados a partir da base de conhecimento da Wikipédia, cujo output são ficheiros .html que apresentam os dados obtidos de uma forma agradável e bem estruturada.
\end{resumo}


\newpage
\section{Agradecimentos}
O presente relatório descreve todo o processo de desenvolvimento, tomadas de decisão e resultados obtidos na resolução do primeiro projeto prático proposto no âmbito da Unidade Curricular de Processamento de Linguagens.
O enunciado aqui resolvido aborda o tema referente ao processamento de informação extraída da Wikipédia - variante 1. O produto final obtido consiste num processador de ficheiros .xml, descarregados a partir da base de conhecimento da Wikipédia, cujo output são ficheiros .html que apresentam os dados obtidos de uma forma agradável e bem estruturada.


\newpage
\section{Introdução}
\par Texto.
\par Texto.


\subsection{Descrição do Problema e Implementação}
\par Texto.
\par Texto.



\begin{description}
  \item[Título] {\tt <title>.*</title>}
\end{description}

\subsection{Estados da Aplicação}
Para o desenvolvimento do analisador léxico recorremos a um autómato constituído por vários estados: \emph{PAGE, TITLE, REVISION, DATE, AUTHOR, LINKINT, LINKEXT, SECTION1, SECTION2, SECTION3}.\\

{\bf Estado INITIAL}

\par Neste estado procura-se pela {\it tag} {\tt <page>} e assim que ela é encontrada entramos no estado PAGE e é também inicializada a estrutura de dados \emph{Pagina pag}, que irá ser abordada mais adiante neste relatório.\\

{\tt ''<page>'' \{ \par \hspace{1cm} BEGIN PAGE; pag = inicializaPagina(); \par \}\\}


\newpage
{\bf Estado PAGE}


\par Quando é encontrada a {\it tag} {\tt </page>} é criada uma página {\it HTML} para a página analisada: é inserido o título da página numa lista ligada para no final ser criado um índice de todas as páginas processadas. No final é chamado o estado INITIAL para verificar, caso exista, uma nova página.\\
 
{\tt 
<PAGE>\{ \par
\hspace{1cm}	''<title>'' \hspace{1.65cm} BEGIN TITLE; \par
\hspace{1cm}    ''<revision>'' \hspace{1.05cm} BEGIN REVISION; \par
\hspace{1cm}    .|[\escape{n}\escape{t}] \hspace{1.75cm} ; \par
\hspace{1cm}	''</page>'' \hspace{1.65cm} criaFicheiroHTML(pag); insereTituloIndice(lpags,pag->titulo); BEGIN INITIAL; \par
\}\\}

{\bf Estado TITLE}

\par Aqui encontramos o título da página que estamos a analisar e guardamo-lo na nossa estrutura. 

\par De modo a evitar captar páginas que contém como título {\emph Categoria:Astronomia}, por exemplo, e que são irrelevantes pois apenas listam vários temas relacionados com Astronomia, usamos a expressão {\tt .*:.*} que se for coicidente descarta essa página, ou seja, vai verificar uma nova página (chamada do estado INITIAL).\\

{\tt <TITLE>\{ \par
\hspace{1cm}	.*:.*	\hspace{2.0cm} BEGIN INITIAL; \par
\hspace{1cm}	[$ \wedge $<]*	\hspace{1.95cm} insereTitulo(pag,yytext); \par
\hspace{1cm}	''</title>'' 	\hspace{1.0cm} BEGIN PAGE; \par
\}\\}
  



{\bf Estado AUTHOR}
\par Tal como com a data da última revisão, o valor obtido pela expressão é guardado e depois volta ao estado REVISION.\\

{\tt <AUTHOR>\{ \par
\hspace{1cm}	[$ \wedge $<]*	\hspace{1.8cm}		insereAutor(pag,yytext); \par
\hspace{1cm}	"</username>"	\hspace{0.8cm}BEGIN REVISION; \par
\}\\}



{\bf Estados SECTION1, SECTION2, SECTION3 e SECTION4}

\par Nestes estados são obtidos os nomes das secções existentes em cada página da Wikipédia. Como os estados são bastante parecidos decidimos agrupá-los. Existe ainda uma expressão que deteta quando uma secção está errada, voltando assim ao estado REVISION. 


\par Cada secção originada é inserida numa lista de secções onde é identificado o tipo de secção.
Finalmente, detetada a expressão regular final, volta ao estado REVISION.\\


{\tt <SECTION1>\{ \par
\hspace{1cm}	.|[\escape{n}\escape{t}]	\hspace{0.5cm}	BEGIN REVISION; \par
\hspace{1cm}	[$ \wedge $=\escape{n}]*	\hspace{0.5cm}		insereSeccao(0,pag,yytext); \par
\hspace{1cm}	== 	\hspace{1.55cm}		BEGIN REVISION;   \par
\}\\} 

{\tt<SECTION2>\{ \par
\hspace{1cm}	[$ \wedge $=\&]+		\hspace{0.8cm}	insereSeccao(1,pag,yytext); \par
\hspace{1cm}	.|[\escape{n}\escape{t}]	\hspace{0.5cm}	; \par
\hspace{1cm}	=== 	\hspace{1.5cm}		BEGIN REVISION; \par
\}\\}

{\tt<SECTION3>\{ \par
\hspace{1cm}	[$ \wedge $=]*		\hspace{0.5cm}	insereSeccao(2,pag,yytext); \par
\hspace{1cm}	==== 	\hspace{0.8cm}	BEGIN REVISION; \par
\}\\}

{\tt<SECTION4>\{ \par
\hspace{1cm}	[$ \wedge $=]*	\hspace{0.5cm}		insereSeccao(3,pag,yytext); \par
\hspace{1cm}	===== 	\hspace{0.5cm}	BEGIN REVISION; \par
\}\\}

{\bf Estado LINKEXT}

Aqui é obtido o URL total do link externo até encontrar um espaço. Depois de inserido na lista de links externos da página, o estado REVISION é chamado.\\

{\tt <LINKEXT>\{ \par
\hspace{1.0cm}	[a-zA-Z0-9/.-\_?=\&;+-\%]+	\hspace{0.5cm} insereLinkExt(pag,yytext); BEGIN REVISION;  \par
\}\\}

\subsection{Módulos da Aplicação}

A aplicação {\emph PLIKIPÉDIA} desenvolvida tem por base os seguintes módulos:\\

{\bf parserXML.fl} Encontra-se o código fonte para fazer a análise léxica aos ficheiros XML.\\

{\bf linkedlist.h} Contém o código fonte das listas ligadas genéricas e suas funções.\\

{\bf auxstruct.h} Tem o código para inserir dados na estrutura de dados Pagina.\\

{\bf htmlpage.h} É onde se encontra o código que gera as páginas .html de cada página da Wikipédia, além do índice de títulos lidos.\\

{\bf makefile} Ferramenta com a configuração de compilação dos ficheiros acima descritos.\\


\subsection{Estruturas de Dados}

Como verificamos\\


{\tt
typedef struct sPagina \{ \par
\hspace{1cm}	char* titulo; \par
\hspace{1cm}	char* data; \par
\hspace{1cm}	char* autor; \par
\hspace{1cm}	LinkedList seccoes; \par
\hspace{1cm}	LinkedList linkint; \par
\hspace{1cm}	LinkedList linkext; \par
\} *Pagina, NPagina; \\
}

Depois de criada a página HTML referente à última página abordada a estrutura de dados é limpa e inicializada para recolher os novos dados de outra página, caso estes existam.\\

\newpage
\section{HTML}

\par Cena









\subsection{Grupo}
\par Por fim, a última tab da barra de navegação rápida é a que apresenta os elementos do grupo que realizou este projeto prático.
\par
\begin{figure}[H]
\includegraphics[scale=0.41]{images/grupo.png}
\end{figure}
\newpage

\section{Conclusões}
\par asdasdasd


\end{document}
