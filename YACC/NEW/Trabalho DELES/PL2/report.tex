\documentclass[12pt]{article}\usepackage{a4wide}\usepackage[portuges]{babel}\usepackage[utf8]{inputenc}\usepackage{hyperref}\usepackage{graphicx}\usepackage{enumerate}\usepackage{multirow}\usepackage{multicol}\begin{document}\begin{titlepage}\newcommand{\\HRule}{\\rule{\\linewidth}{0.5mm}}\center\textsc{\\LARGE Universidade do Minho}\\[1.5cm]\textsc{\\LARGE Report 2007}\\[1.5cm]\textsc{\\large Vamos escrever relatórios}\\[0.5cm]\HRule \\[0.4cm]{ \huge fseries Report 2007}\\[0.4cm]\HRule \[1.5cm]egin{minipage}{0.4\textwidth}\begin{flushleft} \\large\emph{Realizado por:}\\Serafim Pinto \textsc{61056} a61056@alunos.uminho.pt\\Daniel Araújo \textsc{61058} a61058@alunos.uminho.pt\\Daniel Carvalho \textsc{61008} a61008@alunos.uminho.pt\\\end{flushleft}\end{minipage}\begin{minipage}{0.4\textwidth}\begin{flushright} \\large\end{flushright}\end{minipage}\\[4cm]{\large \today}\\[3cm]\includegraphics{UM}\[5cm]\vfill\end{titlepage}\newpage\begin{abstract}<p> Este relatório descreve o processo de desenvolvimento e o resultado obtido, como consequência da resolução do enunciado do trabalho prático número dois da Unidade Curricular Processamento de Linguagens. O enunciado aqui resolvido é sobre o "Report 2007: vamos escrever relatórios". Este documento trata de analisar e explicar os objectivos deste segundo trabalho. Assim, falaremos das decisões tomadas, os principais obstáculos encontrados e os resultados obtidos. Para além dos conhecimentos que já eram necessários para o primeiro projecto, foi necessário demonstrar conhecimentos sobre ferramentas <i>Flex e Yacc</i> , conhecimentos estes que se mostraram muito úteis e interessantes. </p>\end{abstract}\newpage\tableofcontents\end{document}